Das folgende Theorem war schon im antiken Ägypten und Babylonien bekannt.
\begin{thm}[Satz von Thales]
Konstruiert man ein Dreieck aus den beiden Endpunkten des Durchmessers eines Halbkreises (Thaleskreis) und einem weiteren Punkt dieses Halbkreises, so erhält man immer ein rechtwinkliges Dreieck.
\end{thm}

Ich mag die Bücher \cite[]{kafka2015prozess} und \cite{AC02615918}. \textcite[S. 1]{kafka2015prozess} schreibt im Jahr 1914:
\begin{quote}
Jemand mußte Josef K. verleumdet haben, denn ohne daß er etwas Böses getan hätte, wurde er eines Morgens verhaftet.
\end{quote}