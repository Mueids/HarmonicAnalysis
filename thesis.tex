\documentclass[10pt,a4paper]{article}

% Encodings und Fonts -----------------------------------
\usepackage[T1]{fontenc}
\usepackage[utf8]{inputenc}
\usepackage{microtype}
% -------------------------------------------------------

% Spracheinstellungen -----------------------------------
% Verwende option `english` für englische Texte
\usepackage[ngerman]{babel}
\usepackage{csquotes}
% -------------------------------------------------------

% Bibliographie -----------------------------------------
\usepackage[backend=biber,
			style=numeric,
			isbn=false,
			doi=false
			]{biblatex}
\addbibresource{./content/references.bib}
% -------------------------------------------------------

% Typographie -------------------------------------------
\renewcommand{\theenumi}{(\roman{enumi})}
\usepackage{graphicx}
\usepackage{setspace}
% -------------------------------------------------------

% Mathe Pakete, Einstellungen und Commands --------------
\usepackage{amsmath}
\usepackage{amsfonts}
\usepackage{amssymb}
\usepackage{amsthm}

\numberwithin{equation}{section}

\DeclareMathOperator{\N}{\mathbb{N}}
\DeclareMathOperator{\Z}{\mathbb{Z}}
\DeclareMathOperator{\Q}{\mathbb{Q}}
\DeclareMathOperator{\R}{\mathbb{R}}
\DeclareMathOperator{\F}{\mathbb{F}}
\DeclareMathOperator{\Aut}{Aut}
\DeclareMathOperator{\id}{id}
\DeclareMathOperator{\kernel}{ker}
\DeclareMathOperator{\im}{im}
% -------------------------------------------------------

% Theorem-Umgebungen ------------------------------------
\usepackage{thmtools}

\declaretheorem[
	name=Theorem,
	numberwithin=section
	]{thm}
\declaretheorem[
	name=Lemma,
	sibling=thm,
	]{lem}
\declaretheorem[
	name=Proposition,
	sibling=thm,
	]{prop}
\declaretheorem[
	name=Korollar,
	sibling=thm,
	]{cor}

\declaretheorem[
	name=Definition,
	style=definition,
	numbered=no,
	]{defin}

\declaretheorem[
	name=Bemerkung,
	style=remark,
	numbered=no
	]{rem}
\declaretheorem[
	name=Beispiel,
	style=remark,
	numbered=no
	]{exam}
% -------------------------------------------------------

% Eigene Commands ---------------------------------------
\newcommand{\dummy}{Donaudampfschifffahrtsgesellschaft}
% -------------------------------------------------------

% Hyperref (immer am Schluss der Präambel!) -------------
\usepackage[pdftex]{hyperref}
\hypersetup{pdftitle={Beweis des Satz von Thales},
			pdfauthor={Maxime Musterfrau}}
% -------------------------------------------------------

%%%%%%%%%%%%%%%%%%%%%%%%%%%%%%%%%%%%%%%%%%%%%%%%%%%%%%%%%
%%% Beginn des Dokuments
%%%%%%%%%%%%%%%%%%%%%%%%%%%%%%%%%%%%%%%%%%%%%%%%%%%%%%%%%
\begin{document}

% Titelei -----------------------------------------------
%!TEX root = ../thesis.tex
\begin{titlepage}
%\vspace*{-2cm}  % bei Verwendung von vmargin.sty
\begin{flushright}
    \includegraphics[width=8cm]{Uni_Logo_2016_SW}
\end{flushright}
%\vspace{1cm}

\begin{center}  % Diplomarbeit ODER Magisterarbeit ODER Dissertation
    \Huge{\textbf{\textsf{\MakeUppercase{
        1. Bachelorarbeit
    }}}}
    \vspace{2cm}

    \large{\textsf{  % Diplomarbeit ODER Magisterarbeit ODER Dissertation
                     % (Dies ist erst die Ueberschrift!)
        Titel der Bachelorarbeit
    }}
    \vspace{.1cm}

    \LARGE{\textsf{  Beweis des Satz von Thales
    }}
    \vspace{3cm}

    \large{\textsf{  % Verfasserin ODER Verfasser (Ueberschrift)
        Verfasser
    }}

    \Large{\textsf{  Maxime Musterfrau
    }}
    \vspace{3cm}

    \large{\textsf{
        angestrebter akademischer Grad  % (Ueberschrift)
    }}

    \Large{\textsf{  % Magistra ODER Magister ODER Doktorin ODER Doktor
                     % ACHTUNG: Kuerzel "Mag.a" oder "Dr.in" nicht zulaessig
        Bachelor of Science (BSc.)
    }}
\end{center}
\vspace{2cm}

\noindent\textsf{Wien, im Monat Juni 2013}  % <<<<< ORT, MONAT UND JAHR EINTRAGEN
\vfill

\noindent\begin{tabular}{@{}ll}
\textsf{Studienkennzahl lt.\ Studienblatt:}
&
\textsf{A 033621}  % <<<<< STUDIENKENNZAHL EINTRAGEN
\\
    % BEI DISSERTATIONEN:
%\textsf{Dissertationsgebiet lt. Studienblatt:}
    % ANSONSTEN:
\textsf{Studienrichtung lt.\ Studienblatt:}
&
\textsf{Mathematik}  % <<<<< DISSGEBIET/STUDIENRICHTUNG EINTRAGEN
\\
% Betreuerin ODER Betreuer:
\textsf{Betreuer: }
&
\textsf{ao. Univ.-Prof. Dr. Max Mustermann}  % <<<<< NAME EINTRAGEN
\end{tabular}

\end{titlepage}


\newpage

\thispagestyle{empty}

%!TEX root = ../thesis.tex
\section*{Abriss}
Dies ist ein Template für Abschlussarbeiten an der Fakultät für Mathematik der Universität Wien
\vspace{1.5cm}
\section*{Abstract}
This is a template for theses at the Faculty of Mathematics of the University of Vienna.

\newpage

\thispagestyle{empty}
\tableofcontents

\newpage
% -------------------------------------------------------

\renewcommand{\thepage}{ \arabic{page} }

\setcounter{page}{1}
\onehalfspacing

\section{Abschnitt}
%!TEX root = ../thesis.tex
Im folgenden Abschnitt setzen wir die Existenz und elementare Eigenschaften der ganzen Zahlen $\Z$ voraus.

\begin{thm}
  Die Relation auf der Menge $\Z \times \Z \setminus \lbrace 0 \rbrace$, die durch
  \[ (a, b) \sim (c, d) :\Leftrightarrow ad = cb \]
  definiert ist, ist eine Äquivalenzrelation.
\end{thm} 

\begin{proof}
  Seien $a$ und $b \neq 0$ ganze Zahlen, dann gilt $ab = ab$ und daher $(a, b) \sim (a, b)$, d.~h. $\sim$ ist reflexiv. Seien nun $c$ und $d \neq 0$ weitere ganze Zahlen, für die $(a, b) \sim (c, d)$ gilt, dann folgt nach der Definition der Relation die Identität $0 = ad - cb = da - bc$ und daher auch $(c, d) \sim (a, b)$.
  
  Sind schließlich $e$ und $f \neq 0$ weitere ganze Zahlen, für die $(c, d) \sim (e, f)$ gilt, dann muss die Gleichheit $cf = ed$ gelten.
  Da gleichzeitig $ad = cb$ erfüllt ist, folgert man
  \begin{equation}\label{eq:rationals}
    0 = ad - cb = (ad - cb) e = ade - cbe = acf - cbe = c (af - eb).
  \end{equation}
Angenommen $c = 0$, dann müssen wegen $b, f \neq 0$ und $ad = cb = 0$ sowie $cf = ed = 0$ die Zahlen $a$ und $e$ gleich null sein. Man folgert $af = 0 = eb$ und $(a, b) \sim (e, f)$.
  
  Andererseits angenommen, dass $c \neq 0$, dann folgt aus der Gleichung~\ref{eq:rationals} die Identität $af - eb = 0$, da die ganzen Zahlen ein Integritätsbereich sind. Man schließt $(a, b) \sim (e, f)$.
\end{proof}

\begin{thm}\label{thm:rationals}
  Die Äquivalenzklassen von $\Z \times \Z \setminus \{0\}$ bzgl. der Relation $\sim$ aus dem obigen Theorem bilden bezüglich der Oprationen 
  \[ [a, b] + [c, d] := [ad + cb, bd] \]
  und
  \[ [a, b] \cdot [c, d] := [ac, bd] \]
  einen Körper.
\end{thm}

Das folgende Theorem war schon im antiken Ägypten und Babylonien bekannt.
\begin{thm}[Satz von Thales]
Konstruiert man ein Dreieck aus den beiden Endpunkten des Durchmessers eines Halbkreises (Thaleskreis) und einem weiteren Punkt dieses Halbkreises, so erhält man immer ein rechtwinkliges Dreieck.
\end{thm}

Ich mag die Bücher \cite[]{kafka2015prozess} und \cite{AC02615918}. \textcite[S. 1]{kafka2015prozess} schreibt im Jahr 1914:
\begin{quote}
Jemand mußte Josef K. verleumdet haben, denn ohne daß er etwas Böses getan hätte, wurde er eines Morgens verhaftet.
\end{quote}


\vspace{\fill}
\printbibliography{}
\end{document}
