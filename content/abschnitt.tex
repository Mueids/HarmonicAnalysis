%!TEX root = ../thesis.tex
Im folgenden Abschnitt setzen wir die Existenz und elementare Eigenschaften der ganzen Zahlen $\Z$ voraus.

\begin{thm}
  Die Relation auf der Menge $\Z \times \Z \setminus \lbrace 0 \rbrace$, die durch
  \[ (a, b) \sim (c, d) :\Leftrightarrow ad = cb \]
  definiert ist, ist eine Äquivalenzrelation.
\end{thm} 

\begin{proof}
  Seien $a$ und $b \neq 0$ ganze Zahlen, dann gilt $ab = ab$ und daher $(a, b) \sim (a, b)$, d.~h. $\sim$ ist reflexiv. Seien nun $c$ und $d \neq 0$ weitere ganze Zahlen, für die $(a, b) \sim (c, d)$ gilt, dann folgt nach der Definition der Relation die Identität $0 = ad - cb = da - bc$ und daher auch $(c, d) \sim (a, b)$.
  
  Sind schließlich $e$ und $f \neq 0$ weitere ganze Zahlen, für die $(c, d) \sim (e, f)$ gilt, dann muss die Gleichheit $cf = ed$ gelten.
  Da gleichzeitig $ad = cb$ erfüllt ist, folgert man
  \begin{equation}\label{eq:rationals}
    0 = ad - cb = (ad - cb) e = ade - cbe = acf - cbe = c (af - eb).
  \end{equation}
Angenommen $c = 0$, dann müssen wegen $b, f \neq 0$ und $ad = cb = 0$ sowie $cf = ed = 0$ die Zahlen $a$ und $e$ gleich null sein. Man folgert $af = 0 = eb$ und $(a, b) \sim (e, f)$.
  
  Andererseits angenommen, dass $c \neq 0$, dann folgt aus der Gleichung~\ref{eq:rationals} die Identität $af - eb = 0$, da die ganzen Zahlen ein Integritätsbereich sind. Man schließt $(a, b) \sim (e, f)$.
\end{proof}

\begin{thm}\label{thm:rationals}
  Die Äquivalenzklassen von $\Z \times \Z \setminus \{0\}$ bzgl. der Relation $\sim$ aus dem obigen Theorem bilden bezüglich der Oprationen 
  \[ [a, b] + [c, d] := [ad + cb, bd] \]
  und
  \[ [a, b] \cdot [c, d] := [ac, bd] \]
  einen Körper.
\end{thm}

Das folgende Theorem war schon im antiken Ägypten und Babylonien bekannt.
\begin{thm}[Satz von Thales]
Konstruiert man ein Dreieck aus den beiden Endpunkten des Durchmessers eines Halbkreises (Thaleskreis) und einem weiteren Punkt dieses Halbkreises, so erhält man immer ein rechtwinkliges Dreieck.
\end{thm}

Ich mag die Bücher \cite[]{kafka2015prozess} und \cite{AC02615918}. \textcite[S. 1]{kafka2015prozess} schreibt im Jahr 1914:
\begin{quote}
Jemand mußte Josef K. verleumdet haben, denn ohne daß er etwas Böses getan hätte, wurde er eines Morgens verhaftet.
\end{quote}
